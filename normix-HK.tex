\documentclass[11pt]{article}
\usepackage{texab}
\usepackage{parskip}

\begin{document}

\title{Quantiles of normal mixtures}
 \author{Hans R. K\"unsch, ETH Zurich}
\date{March 13, 2010}

\maketitle
Let $F$ be the distribution function of a normal mixture
$$F(x) = \sum_{i=1}^n w_i \Phi\left(\frac{x-\mu_i}{\sigma_i}\right).$$
We want to compute the $\alpha$-quantile $q$ of $F$. The algorithm that I
propose here computes recursively the $\alpha$-quantile $q_k$ of
$$F_k(x) = \sum_{i=1}^n w_i \Phi_k\left(\frac{x-\mu_i}{\sigma_i}\right)$$
where 
$$\Phi_k(x) = \frac{l}{2^k-1} \quad (z(k,l) \leq x < z(k,l+1); \
l=0,1,\ldots,2^k),$$ 
$z(k,0)=-\infty$, $z(k,2^k)=+\infty$ and
$$z(k,l) = \Phi^{-1}(l2^{-k}) \quad (l=1,2, \ldots , 2^k-1).$$
$\Phi_k$ is a balanced discrete approximation of $\Phi$. It is
clear that 
$$\sup_x |F_k(x)-F(x)| \leq 2^{-k}.$$
Therefore we have a direct control on the error $|F(q_k)-\alpha|$.

We start the recursion at $k=0$: $q_0$ is simply
the weighted $\alpha$-quantile of the means $\mu_i$. I assume that at step
$k-1$ $q_{k-1}$ has been computed and for each $i$ we have stored 
$$l_i = \Phi_{k-1}\left(\frac{q_{k-1}-\mu_i}{\sigma_i}\right) (2^{k-1}
-1).$$

The recursion consists of two parts. In the first part, we compute
$F_{k}(q_{k-1}) - F_{k-1}(q_{k-1})$. This is easy, as we only need to
check for each $i$ whether
$$z(k,2l_i+1) < \frac{q_{k-1} - \mu_i}{\sigma_i} \textrm{ or } z(k,2l_i+1) \geq
\frac{q_{k-1} - \mu_i}{\sigma_i},$$
and we can give a formula for the difference (not done here). Then we know
if $q_k > q_{k-1}$ or $q_k < q_{k-1}$. In addition, we replace $l_i$ by $2
l_i$ or by $2 l_i+1$, respectively, so that the $l_i$ are the values of
$(2^k-1)\Phi_k((q_{k-1}-\mu_i)/\sigma_i)$. 

The second part consist in modifying $q_{k-1}$. Let us assume $q_k >
q_{k-1}$. We then have to determine how far we have to go to the right.
Note that the next point where $F_k$ jumps is the smallest of the values 
$\mu_i + \sigma_i z(k,l_i+1)$. Hence we create a vector which contains
these values in ascending order. At the first value of this vector, 
$F_k$ increases by $1/(2^k-1)$ times the weight of the corresponding component,
which I denote by $i$. We then increase this $l_i$ by one, remove 
the first element of the vector and insert instead the new value 
$\mu_i + \sigma_i z(k,l_i+1)$ at the correct position. We continue this
until we find the point where $F_k > \alpha$ for the first time
which is the value $q_k$. The values $l_i$ have also been adjusted
during the procedure. 

I believe that this algorithm is fast for arbirtrary values of 
the parameters, in particular if $n$ is large. If we need the
quantiles for many different $\alpha$, maybe something else will
be more efficient. 

\end{document}


%%% Local Variables: 
%%% mode: latex
%%% TeX-master: t
%%% End: 
